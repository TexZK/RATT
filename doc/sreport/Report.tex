% VDE Template for EUSAR Papers
% Provided by Barbara Lang und Siegmar Lampe
% University of Bremen, January 2002
% English version by Jens Fischer
% German Aerospace Center (DLR), December 2005
% Additional modifications by Matthias Wei{\ss}
% FGAN, January 2009

%-----------------------------------------------------------------------------
% Type of publication
\documentclass[a4paper,10pt]{article}
%-----------------------------------------------------------------------------
% Other packets: Most packets may be downloaded from www.dante.de and
% "tcilatex.tex" can be found at (December 2005):
% http://www.mackichan.com/techtalk/v30/UsingFloat.htm
% Not all packets are necessarily needed:
\usepackage[T1]{fontenc}
\usepackage[latin1]{inputenc}
%\usepackage{ngerman} % in german language if required
\usepackage[nooneline,bf]{caption} % Figure descriptions from left margin
\usepackage{times}
\usepackage{multicol}
\usepackage{amsmath}
\usepackage{amssymb}
\usepackage[dvips]{graphicx}
\usepackage{epsfig}
\usepackage{mdwlist}
\input{tcilatex}
%-----------------------------------------------------------------------------
% Page Setup
\textheight24cm \textwidth17cm \columnsep6mm
\oddsidemargin-5mm                 % depending on print drivers!
\evensidemargin-5mm                % required margin size: 2cm
\headheight0cm \headsep0cm \topmargin0cm \parindent0cm
\pagestyle{empty}                  % delete footer and header
%----------------------------------------------------------------------------
% Environment definitions
\newenvironment*{mytitle}{\begin{LARGE}\bf}{\end{LARGE}\\}%
\newenvironment*{mysubtitle}{\bf}{\\[1.5ex]}%
\newenvironment*{myabstract}{\begin{Large}\bf}{\end{Large}\\[2.5ex]}%
%-----------------------------------------------------------------------------
% Using Pictures and tables:
% - Instead "table" write "tablehere" without parameters
% - Instead "figure" write "figurehere " without parameters
% - Please insert a blank line before and after \begin{figuerhere} ... \end{figurehere}
%
% CAUTION:   The first reference to a figure/table in the text should be formatted fat.
%
%\begin{figurehere}
%  \centering
%  \includegraphics[width=8cm, height=4cm]{./eps/placeholder.eps}
%  \caption{Some single-column figure caption.}
%  \label{fig:myfigure1}
%\end{figurehere}
%
%\begin{figure*}[t]
%  \centering
%  \includegraphics[width=16cm, height=4cm]{./eps/placeholder.eps}
%  \caption{Some wide-figure caption.}
%  \label{fig:myfigure2}
%\end{figure*}

\makeatletter
\newenvironment{tablehere}{\def\@captype{table}}{}
\newenvironment{figurehere}{\def\@captype{figure}\vspace{2ex}}{\vspace{2ex}}
\makeatother

%%%%%%%%%%%%%%%%%%%%%%%%%%%%%%%%%%%%%%%%%%%%%%%%%%%%%%%%%%%%%%%%%%%%%%%%%%%%%%
\begin{document}

\newcommand{\TODO}{\textbf{TODO\dots\ }}
\newcommand{\CITEME}{\textbf{[CITEME]}}
\newcommand{\INSFIG}{\textbf{Figure PLACEHOLDER}}


\begin{mytitle}RATT	- Relatiely Accurate TurnTable\end{mytitle}
\begin{mysubtitle}Relative turntable motion detection with mouse sensors\end{mysubtitle}
% Please do not insert a line here
\\
Zoppi Andrea\\
Matr. 765662, (andrea.zoppi@mail.polimi.it)\\
\begin{flushright}
\emph{Report for the master course of Embedded Systems}\\
\emph{Reviser: PhD. Patrick Bellasi (bellasi@elet.polimi.it)}
\end{flushright}

Received: <MONTH>, <DAY> 2012\\
\hspace{10ex}

\begin{myabstract} Abstract \end{myabstract}
Cheap jogwheel encoders for emulated DJ turntables are often inaccurate, due to
the low CPR in the order of some tens. Precise encoders are rather expensive and
are not convenient when detecting very fast rotations, because they become too
much accurate for the purpose.

The presented research tries to improve the detection of at least small and slow
relative rotations of the jogwheel, by employing a cheap COTS mouse sensor, and
keep the absolute position or fast rotation with the classic cheap optical encoder.

A very crude HID demoboard was developed, so that some simple tests were done.
An extended proposal is also described, in order to achieve better performance
and more features, to match those of a commercial DJ controller.


\vspace{4ex}	% Please do not remove or reduce this space here.
\begin{multicols}{2}


%%%%%%%%%%%%%%%%%%%%%%%%%%%%%%%%%%%%%%%%%%%%%%%%%%%%%%%%%%%%%%%%%%%%%%%%%%%%%
\section{Introduction}

In the latest years, the market of digital-DJ related products grew considerably.
By the way, cheap digtital turntable emulation is still tricky, because the design
of a cheap yet accurate jogwheel is still a challenge even with the technology
available today.

It is true that state-of-the-art processing units are very fast, but there are
still issues such as those related to protocol latency, precise and fast plate
motion detection, motion samples interpolation, and so on.

These issues are not a big problem for the average DJ, but they arise when requiring
a higher performance (e.g. \emph{scratch}) while keeping the costs low.

The proposed approach is based on COTS components called \emph{optical motion/mouse
sensors}, which can provide a very good accuracy when detecting small local motions,
which is a behavior difficult to obtain with cheap encoders.

A simplified verison of the proposal was developed on a crude prototype, just to
check if it is worth at least for the average DJ -- the most demanding ones do
not care about the price of products, and still rely on timecoded vinyl emulation
even tough high CPR optical encoders are available at the same overall price.


%%%%%%%%%%%%%%%%%%%%%%%%%%%%%%%%%%%%%%%%%%%%%%%%%%%%%%%%%%%%%%%%%%%%%%%%%%%%%
\section{Current market}

The common commercial approaches can be divided into two groups: jogwheels based
on optical encoders, and reuse of vinyl turntables (or CD players) as if they were
digital jogwheels. These two technologies will be described in the following, showing
their pros and cons.

There exist also some other ways to emulate turntables, which are currently still
in a niche. For example, there are some touchscreen-based turntable controllers
\CITEME, which follow the market wave of touchscreen devices.
There are also some evolutions of the optical jogwheels, which are motorized \CITEME
and thus more suitable for professionals, but rather expensive.


%-----------------------------------------------------------------------------
\subsection{Optical encoder jogwheel controllers}

The most common technology for turntable emulation is based on optical encoders.
An optical encoder is a device which detects motion by counting the number of steps
an evenly-marked wheel performs. It is found in almost all purely digital DJ
\emph{controllers}, which in this context are referred to those remote digital
devices used by the DJ to control the user application. Some examples of commercial
controllers with jogwheels can be found in \CITEME.

Common jogwheels have a resolution (\emph{CPR, Counts Per Revolution}) in the order of
tens, thus are not suitable for \emph{scratching}\CITEME, and are usually addressed
only in coarse track navigation, or \emph{bending}\CITEME.
Even a resolution in the order of some hundreds cannot be enough for scratch or precise
motion tracking. For example, with a 720 CPR encoder it is possible to detect only
motions of half degrees, that for a 12 inches wide wheel is still low -- keep in mind
that a vinyl spins at roughly 150 degrees per second, and good sampling should require
at least 1500 samples per second to track it decently enough.


\paragraph{System architecture}
The basic architecture of these controllers is shown is \textbf{Figure \CITEME}.
The controller commonly has a set of input devices -- buttons, knobs,
sliders, \textit{etc.} -- so that the user can \emph{map}\footnote{
	Assign a triggered action to an input event
} these inputs to some software parameters, such as the play/stop events, or the desired
volume level.

A special kind of input device focused throughout this work is the optical encoder,
\TODO which will be described in depth later.

Commonly, there is a also set of output devices -- LEDs, displays -- so that the
user's sight should not always keep switching switched between the computer monitor
and the controller to see what is going on.

All these devices are managed by a MCU, which detects their changes, and generates
meaningful messages to be sent to the DJ software, or receives messages from the latter.

The communication between the controller and the software is often performed through
an USB bus with HID/USB or MIDI/USB protocols, but some controllers still rely on
the plain old MIDI port (see Section~\CITEME).


\paragraph{Wheel architecture}
The wheel is emulated with a so-called \emph{jogwheel} \INSFIG. This is a disc whose full
rotation is divided into equalli-spaced angle slices. Each slice is assigned a code.

Usually, the code is marked on the wheel with holes aligned on circles (see \INSFIG),
so that holes can be detected by light-detector sensors mounted on the chassis.
The light-detector sensor is almost always made with a LED which points towards the
disc, and on the other side the light is detected by a fast phototransistor.
These light sensors are positioned so that they can detect one and only one code per slice.

The code is either abolute or relative. Absolute encoders assign a unique code to each
slice, so that it is always possible to know the current wheel angle by just reading the
light-detector sensors outputs. Due to the need to have a high number of bits, the number
of holes can also grow exponentially (usually as the power of two), and the production of
precisely aligned marks and sensors is expensive -- misaligned ones can provide
misdetections of the angle, even with robust codes such as the \emph{Grey code} \CITEME.

Instead, relative encoders just need the two least significant bits of an absolute code,
thus cheaper to manufacture. On the other hand, it is not possible to know the absolute
rotation without any additional bits. This is why there is often a mark which signals
a full revolution been performed, and needs an additional flag bit. The particular
subset of the Grey code used for the relative motion detection is called \emph{quadrature
code}, because only 4 code sequences can be generated by moving to the adjacent wheel
slice \CITEME.


\paragraph{Motion detection}
When the user turns the wheel, the light-detector sensors can convert the sight of
light into the code assigned to the focused disc slice. The digital code is then
triggered by the MCU through some interrupts, and a message containing the motion (or
even the absolute angle) is sent to the user software.

\paragraph{Pros}
\begin{itemize*}
	\item Easy to manufacture
	\item Code detection is inherently digital
	\item Fast code transitions can be processed easily
	\item A cheap MCU can handle jogwheels as well as all the other digital devices
		commonly found in DJ controllers
\end{itemize*}


\paragraph{Cons}
\begin{itemize*}
	\item Small motions have poor resolution with cheap encoders
	\item High resolution encoders are too much expensive for the purpose
\end{itemize*}


%-----------------------------------------------------------------------------
\subsection{Timecoded media turntable emulation}

An alternative way to emulate a turntable in software is to use a \emph{timecoded audio track},
which is an audio stream coded so that the software DSP can read the track position just
by decoding the incoming audio stream.

This technique makes it possible to use existing turntables or CD players to control the
user software, which in turn will emulate the turntable behavior.

The good side of this approach is that a DJ, who already has turntables or CD players,
can keep using them just by buying a sound card with the appropriate audio inputs.
This way the DJ can have almost perfectly the same old feeling, because he is still
using the same equipment.

On the bad side, vinyls and CDs are very sensible to usage, and decay easily. This
makes the timecode unreadable in the ruined parts of the support, thus software cannot always
understand the code. In addition, turntable needles must follow tracks almost perfectly,
or the timecoded signal would degradate at the ADC side, especially the phase component
which is necessary for the purpose, but almost ignored in audio players since the human
ear has poor phase sensitivity.

Another bad point relates to the overall performance. It is true that with this technique
the performance is almost the same of a real vinyl, but the need of an intermiate sound card,
which in turn is often connected through the USB bus, just makes low latencies hard to achieve,
unless the host computer is powerful and well optimised to reach soft-realtime requirements.

Also, a novice DJ would hardly choose this approach, because the overall price of the equipment
can be rather high -- turntables/players + good soundcard + perfect needles + cables +
hi-performance computer can easily exceed \$3000.


\TODO 


\paragraph{System architecture}
\TODO


\paragraph{Wheel architecture}
\TODO


\paragraph{Motion detection}
\TODO


\paragraph{Pros}
\TODO


\paragraph{Cons}
\TODO


%-----------------------------------------------------------------------------

\TODO Add comparison table


%%%%%%%%%%%%%%%%%%%%%%%%%%%%%%%%%%%%%%%%%%%%%%%%%%%%%%%%%%%%%%%%%%%%%%%%%%%%%
\section{Simplified proposed approach}

\TODO


%-----------------------------------------------------------------------------
\subsection{Hardware architecture}

\TODO


\paragraph{Controller board}
\TODO


\paragraph{Sensor board}
\TODO


\paragraph{Timecode preamp board}
\TODO


%-----------------------------------------------------------------------------
\subsection{Firmware architecture}

\TODO


\paragraph{Main module}
\TODO


\paragraph{HID/USB module}
\TODO


\paragraph{LED module}
\TODO


\paragraph{Encoder module}
\TODO


\paragraph{Sensor module}
\TODO


\paragraph{Tasks organization}
\TODO


\paragraph{Remarks}
\TODO


%-----------------------------------------------------------------------------
\subsection{Software architecture}

\TODO


\paragraph{HID/USB connectivity}
\TODO


\paragraph{Device description}
\TODO


\paragraph{Device mapping}
\TODO


%-----------------------------------------------------------------------------
\subsection{Field results}

\TODO


%%%%%%%%%%%%%%%%%%%%%%%%%%%%%%%%%%%%%%%%%%%%%%%%%%%%%%%%%%%%%%%%%%%%%%%%%%%%%
\section{Extended proposed approach}

\TODO


%-----------------------------------------------------------------------------
\subsection{Hardware architecture}

\TODO


%-----------------------------------------------------------------------------
\subsection{Firmware architecture}

\TODO


\paragraph{Main module}
\TODO


\paragraph{HID/USB module}
\TODO


\paragraph{LED module}
\TODO


\paragraph{Encoder module}
\TODO


\paragraph{Sensor module}
\TODO


\paragraph{Debug module}
\TODO


\paragraph{Tasks organization}
\TODO


\paragraph{Remarks}
\TODO


%%%%%%%%%%%%%%%%%%%%%%%%%%%%%%%%%%%%%%%%%%%%%%%%%%%%%%%%%%%%%%%%%%%%%%%%%%%%%
\section{Appendix A - Brief common MCU survey}

\TODO


%-----------------------------------------------------------------------------
\subsection{Microchip PIC18}

\TODO


\paragraph{Common features}
\TODO


\paragraph{Pros}
\TODO


\paragraph{Cons}
\TODO


%-----------------------------------------------------------------------------
\subsection{Microchip PIC24}

\TODO


\paragraph{Common features}
\TODO


\paragraph{Pros}
\TODO


\paragraph{Cons}
\TODO


%-----------------------------------------------------------------------------
\subsection{Microchip PIC32}

\TODO


\paragraph{Common features}
\TODO


\paragraph{Pros}
\TODO


\paragraph{Cons}
\TODO


%-----------------------------------------------------------------------------
\subsection{Atmel AVR}

\TODO


\paragraph{Common features}
\TODO


\paragraph{Pros}
\TODO


\paragraph{Cons}
\TODO


%-----------------------------------------------------------------------------
\subsection{Atmel AVR32}

\TODO


\paragraph{Common features}
\TODO


\paragraph{Pros}
\TODO


\paragraph{Cons}
\TODO


%-----------------------------------------------------------------------------
\subsection{TI MSP-430}

\TODO


\paragraph{Common features}
\TODO


\paragraph{Pros}
\TODO


\paragraph{Cons}
\TODO


%-----------------------------------------------------------------------------
\subsection{ARM Cortex Mx}

\TODO


\paragraph{Common features}
\TODO


\paragraph{Pros}
\TODO


\paragraph{Cons}
\TODO


%%%%%%%%%%%%%%%%%%%%%%%%%%%%%%%%%%%%%%%%%%%%%%%%%%%%%%%%%%%%%%%%%%%%%%%%%%%%%
\section{Appendix B - Optical motion (mouse) sensors}

\TODO


%-----------------------------------------------------------------------------
\subsection{Generic architecture}

\TODO


%-----------------------------------------------------------------------------
\subsection{Operation}

\TODO


%-----------------------------------------------------------------------------
\subsection{Communication}

\TODO


\paragraph{SSP/SPI}
\TODO


\paragraph{I2C}
\TODO


\paragraph{Parallel}
\TODO



%%%%%%%%%%%%%%%%%%%%%%%%%%%%%%%%%%%%%%%%%%%%%%%%%%%%%%%%%%%%%%%%%%%%%%%%%%%%%
\section{Appendix C - Controller communication protocol comparison}

\TODO


%-----------------------------------------------------------------------------
\subsection{HID/USB}

\TODO


\paragraph{Basic protocol}
\TODO


\paragraph{Pros}
\TODO


\paragraph{Cons}
\TODO


%-----------------------------------------------------------------------------
\subsection{MIDI/USB}

\TODO


\paragraph{Basic Protocol}
\TODO


\paragraph{Pros}
\TODO


\paragraph{Cons}
\TODO


%-----------------------------------------------------------------------------
\subsection{Plain MIDI}

\TODO


\paragraph{Basic Protocol}
\TODO


\paragraph{Pros}
\TODO


\paragraph{Cons}
\TODO


%%%%%%%%%%%%%%%%%%%%%%%%%%%%%%%%%%%%%%%%%%%%%%%%%%%%%%%%%%%%%%%%%%%%%%%%%%%%%
% We suggest the use of JabRef for editing your bibliography file (Report.bib)
\bibliographystyle{splncs}
\bibliography{Report}

\end{multicols}
\end{document}
